\documentclass[732,14pt]{studrep}

\RequirePackage{tabularx}
% \RequirePackage{graphicx}

% \usepackage{showframe}

% pygmetize
\RequirePackage{minted}
\RequirePackage{indentfirst}
\RequirePackage{graphicx}
\usemintedstyle{tango} % bw

% \setmainfont{Times New Roman}
\graphicspath{{pics/}}


\setminted{autogobble,mathescape,linenos=true,fontsize=\small,breaklines}
\begin{document}
\thispagestyle{empty}
\begin{center}
Министерство науки и высшего образования Российской Федерации\\
Федеральное государственное бюджетное образовательное учреждение высшего образования
«Иркутский государственный университет»
(ФГБОУ ВО «ИГУ»)\\%[0.5em]
Институт математики и информационных технологий\\%[0.5em]
Кафедра информационных технологий\\%[0em]
\end{center}
\vspace{-1em}
\noindent\begin{tabularx}{\textwidth} {
  >{\raggedright\arraybackslash}X
  >{\raggedright}X }
&
  \begin{center}
    \textbf{УТВЕРЖДАЮ}
  \end{center}%
\vspace{-1em}
  \noindent $\langle$Начальник транспортного цеха$\rangle$

  \noindent $\langle$Джон Иванович Лонг$\rangle$ \\[0.3em]

  \noindent <<\rule{1cm}{0.5pt}>> \hrulefill\ 2023~г. % \\[0.5em]

\vspace{-1em}
  \begin{center}
    М.П.
  \end{center}

\end{tabularx}
% \vspace{1em}
\begin{center}
  \textbf{\large ОТЧЕТ}

  по производственной практике (научно-исследовательской работе)
  % по производственной практике
  % по преддипломной практике   % и т.п.
  %\textbf{ ВЫПУСКНАЯ КВАЛИФИКАЦИОННАЯ РАБОТА\\
%БАКАЛАВРА}
\vspace{1em}

по направлению 02.03.03 -- Математическое обеспечение и администрирование информационных систем

% профиль <<общий>>

\vspace{2em}
% ТЕМА (Можно тему ВКР вписать, или ограничить часть работы в контексте общей задачи ВКР)
 ИССЛЕДОВАНИЕ ОРГАНИЗАЦИОННОЙ СТРУКТУРЫ ОТДЕЛА АСУ ИАПО

\end{center}
\vfill
\noindent\begin{tabularx}{\textwidth} {
  >{\raggedright\arraybackslash}X
  >{\raggedright}X }
&
Студента 4 курса очного отделения\\
группы 02141--ДБ\\
%Фамилия Имя Отчество\\[2em]
$\langle{}$Фамилия Имя Отчество$\rangle$\\[1em]

Руководитель:\\ % Практики
к.~т.~н., доцент\\[0.5em]
\underline{\hspace{3cm}} Черкашин Евгений Александрович

%Защищен с оценкой\\[1em] %\underline{\hspace{3cm}}
%Допущена к защите\\
%Зав.каф., к.~т.~н., доцент\\
%\underline{\hspace{3cm}} Черкашин Евгений\\ \hspace{3cm} Александрович\\[2em]

\end{tabularx}
\vfill
\begin{center}
  Иркутск 2023
\end{center}
\clearpage

\tableofcontents

\chapter*{ВВЕДЕНИЕ}
\label{chap:intro}
% TODO: Add contents line

Реализация компиляторов языков программирования -- одно из основных направлений в области системного программирования, включающего разработку трансляторов (в общем смысле, т.е. и компиляторов и интерпретаторов).  Трансляторы языков программирования относятся к системам порождающего программирования (ПП), т.е. программным системам, задача которых создать исходный код или какой-либо другой объект по некоторому описанию, модели, исходному информационному объекту.  Применение ПП предполагает, что исходный конформационный объект меняется достаточно редко, поэтому имеет смысл повысить производительность целевой системы за счет представления предварительного анализа объекта в виде последовательности инструкций целевого вычислителя, реализующих уже результат анализа.  Сама процедура анализа выполняется один раз транслятором.

\end{document}
