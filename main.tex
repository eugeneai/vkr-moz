\documentclass[732,14pt]{studrep}

\RequirePackage{tabularx}
% \RequirePackage{graphicx}
% Тестирование изменений
% \usepackage{showframe}

% test 

% pygmetize 
\RequirePackage{minted}
\RequirePackage{indentfirst}
\RequirePackage{graphicx}
\usemintedstyle{tango} % bw

% \setmainfont{Times New Roman}
\graphicspath{{pics/}}


\setminted{autogobble,mathescape,linenos=true,fontsize=\small,breaklines}
\begin{document}
\thispagestyle{empty}
\clearpage
 
\tableofcontents

\chapter*{ВВЕДЕНИЕ}
\label{chap:intro}
% TODO: Add contents line

\chapter{Методики прогнозирования состояния береговой зоны}\label{chap:techniques}




Таким образом, для обеспечения возможности проведения научных исследований в области инженерной геологии береговых зон водохранилищ необходимо создать ресурсы хранения информации и вычислительные ресурсы, обеспечивающие продуктивную среду прогнозирования с использованием различных математических моделей. .... модели предназначены для решения задач и ранжируются по видан задач, масштабу исследуемого объекта, размеру интервала времени, степени точности.

\chapter{Проектирование ГИС для прогнозирования береговой зоны}\label{chap:proj}

Анализ предметной области, представленный в Главе~\ref{chap:proj}, показал, что для получения качественно новых результатов научных исследований в инженерной геологии береговых зон внутренних водоемов, необходимо создать ресурсы хранения, преобразования, обеспечения эффективного доступа к этой информации, а также ее визуализации. Данные требования реализуются в виде информационно-вычислительную инфраструктуры, включающей набор сервисов (серверов), взаимодействующих согласно согласованному плану решения задачи. 

\section{Проектирование информационно-вычислительной среды}
\subsection{Функциональные требования}


... Необходимо обеспечить доступ к данным из разных ГИС (QGIS, ArcGIS и др.). Современные ГИС позволяют обеспечивать доступ пространственно-распределенным данным, представленным в специальных форматах, данным баз реляционных данных, а также виртуальных географических сервисов.
\chapter{Реализация подсистем распределенной вычислительной среды}\label{chap:impl}


\section{Тестирование}

\chapter*{ЗАКЛЮЧЕНИЕ}

Выпускная квалификационная работа посвящена проекту развития научного направления Лаборатории инженерной геологии и геоэкологии в аспекте повышения уровня информатизации и обработки данных полевых исследований. В работе решены следующие задачи:
\begin{enumerate}
  \item Проведена формализация предметной области инженерной геологии, относящейся к исследованиям геологических процессов внутренних водоёмов, представлена концептуальная модель,
  \item Разработана информационно-вычислительной инфраструктура поддержки прогнозирования состояния береговой зоны на основе геоинформационной системы и распределенной вычислительной среды,
  \item Предложен вариант архитектуры вычислительной среды, я также реализованы некоторые её подсистемы с использованием современных средств распознавания объектов на изображениях,
  \item Для реализованных подсистем предложены модели данных, предназначенных для хранения информации в процессе её обработки,
  \item Продемонстрирована работоспособность предложенной среды на простом примере прогнозирования.
\end{enumerate}

Рассмотренные в диссертации вопросы преследуют целью формирования в лаборатории вычислительных ресурсов поддержки принятия исследований по анализу опасных геологических процессов. Дальнейшее направление научных исследований и опытно-конструкторских работ имеет смысл продложать по нескольким основным направлениям:
\begin{itemize}
  \item Завершение реализации информационно-вычислительный среды,
  \item Наполнение информационных ресурсов архивными данными и данными современного мониторинга объекта исследования,
  \item Совершенствование методов прогнозирования состояния береговой зоны,
  \item Разработка экспертной системы оценки результатов моделирования с целью формирования реклмендаций по использованию конкретных участков исследуемой береговой зоны.
\end{itemize}



      

\begin{thebibliography}{99}
  

  \bibitem{gruber} Gruber, T. Toward Principles for the Design of Ontologies Used for Knowledge Sharing. International Journal of Human-Computer Studies. 43 (5–6): 907–928. doi:10.1006/ijhc.1995.1081. S2CID 1652449.

Thomas R. Gruber,
Toward principles for the design of ontologies used for knowledge sharing?,
International Journal of Human-Computer Studies,
Vol. 43, No. 5–6,
1995,
Pp. 907-928,
ISSN 1071-5819,
\doi{10.1006/ijhc.1995.1081}, \url{https://www.sciencedirect.com/science/article/pii/S1071581985710816} (дата доступа: 11.05.2024)
\end{thebibliography}




\end{document}
